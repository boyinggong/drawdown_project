Serial correlation, also known as autocorrelation, is the correlation of observations at different time point. Under the wide-sense stationary process assumption, the serial correlation of $X$ between two time point $t$ and $s$ can be measured by the autocorrelation function as follows:

\begin{equation}
R(\tau) = R(s, t) = \frac{E[(X_t-\mu)(X_s-\mu)]}{\sigma^2}
\end{equation}

where $\tau = t-s$.

Serial correlation is often associated with the violation of efficiency market and random walk hypothesis. The literature documenting empirical serial correlation is extensive in the late 1980's\footnote{See Barucci and Emilio (2012, Section 6.5) \cite{barucci2012financial} for a detailed review.}. In stock price, Lo and MacKinlay (1988) \cite{lo1988stock} argue that returns based on the horizon longer than one year show a significant mean reversion, while Poterba and Summers (1988) \cite{poterba1988mean} detect a mean aversion for weekly and monthly returns. Lo and Mavkinlay (1988) \cite{lo1988stock}, Conrad et al. (1991) \cite{conrad1991components} model the security returns using a positively autocorrelated common component, an idiosyncratic component and a white-noise component. More extensively, the serial correlation has been documented in literature on nonsynchronous trading, which means assets are not traded simultaneously \cite{lo1990econometric}. Mech and Timothy (1993) \cite{mech1993portfolio} present evidence that the autocorrelation is associated with the delaying in price adjustment caused by transaction costs. In hedge fund returns, Getmansky, Lo and Makarov (2004) argue that serial correlation is an outcome of illiquidity exposure and smoothed returns, market inefficiencies, time-varying expected returns  and leverage and incentive fees with high water marks.

Assuming a moving average representation of reported returns, Getmansky, Lo and Makarov (2004) \cite{getmansky2004econometric} show that Sharp Ratio (SR) tends to be overstated and the market beta understated. Cesare, Stork and Vries (2014) \cite{di2014risk} use the similar structure to demonstrate that the reported value-at-risk (VaR) and expected shortfall (ES) are always smaller than or equal to their actual values. Thus, the risks of assets are easily underestimated using standard risk measures, and the investment decisions may be misleading. Although based on serial correlation and smoothing feature of hedge fund returns, their models are as well applicable to other assets with autocorrelated returns. 
